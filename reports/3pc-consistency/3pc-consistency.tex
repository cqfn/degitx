\documentclass[acmlarge, screen, nonacm]{acmart}
\usepackage[utf8]{inputenc}
\usepackage{float}
\usepackage{setspace}
\usepackage{pgfplots}
\usepackage{graphicx}
\usepackage{fancyvrb}
\usepackage{listings}
\usepackage{xcolor}
\usepackage{hyperref}
  \hypersetup{colorlinks=true,allcolors=blue!40!black}
\setlength{\topskip}{6pt}
\setlength{\parindent}{0pt} % indent first line
\setlength{\parskip}{6pt} % before par

\title{Three phase commit consistency evaluation for git repository replication}

\author{Ilya Moskvitin}
\email{TODO: put your email address here}

\author{Kirill Chernyavskiy}
\email{g4s8.public@gmail.com}

\acmBooktitle{none}
\acmConference{none}
\editor{none}

\begin{document}

\begin{abstract}
  For git repository replication distributed system we need to use atomic
  commit implementation based on git reference transaction hooks (todo: add reference to previous report).
  There are three most popular implementations: two-phase commit (2PC), three-phase commit (3PC),
  and Paxos commit protocols. 2PC protocol is not really fault tolerant, since it may turn storage
  replicas into inconsistent state in case of transaction manager (TM) failure. Paxos-commit is a fault-tolerant
  version of transaction commit protocol. But it pays for fault-tolerance by sending much more messages via network
  for distributing the resource manager (RM) votes: \texttt{3N - 3} messages for 2PC vs
  \texttt{(N - 1)(2F + 3)} messages for Paxos commit co-located faster version (where N is number of nodes,
  and F is a number of tolerated faults)
  (todo: add references  J. Gray and L. Lamport, “Consensus on transaction commit).
  Three phase commit (3PC) claims to be more fault-tolerant comparing to 2PC but it still have a path to become
  inconsistent. On the other hand, it's much faster comparing to Paxos commit and uses fewer messages to perform a
  transaction (todo: add numbers of messages for each protocol) We need to evaluate this path and estimate how critical
  it could be for managing distributed git repository and what's the probability of this case in real life.
\end{abstract}

\maketitle

\section{Problem}

3PC was introduced by (todo: add authors and initial publication). The motivation of creating 3PC was to fix
a problem with 2PC, which can not recover from some kind of failures. For instance, if a TM receives a ``prepared''
message from all RMs, decides to commit, and sends a ``commit'' message to one of the RM (let's say RM1),
and the crashes, the secondary TM can be started in another network region which can't access RM1 to continue the transaction,
and this secondary TM will not be able to access RM1 and decides to abort the transaction by sending ``abort'' message
to RMs except RM1, then RMs will have different state after the transaction.
The 3PC protocol fixes this problem by introducing new kind of message: ``prepared to commit''. If TM fails before sending
this message, the secondary TM will abort the transaction if at least one RM is not in ``prepared to commit'' state
(todo: describe more verbosly).

\section{Related works}

(todo: describe 3PC papers, paxos-commit about 3PC).

\section{Solution}

3PC could be used as a distributed commit protocol for git transactions. (todo: add diagrams).
\begin{itemize}
  \item Client pushes new changes to coordinator node
  \item Coordinator sends these changes to all RMs
  \item Each RM starts applying these changes and calls reference-transaction hook process with the result (prepared
    in case if transaction was validated and can be applied; aborted otherwise).
  \item Hook calls RM process with a decicion and waits for the signal from RM.
  \item (todo: explain 3PC here)
  \item RM notifies the hook process to continue with decided by TM status
  \item Hook exits with specified status: 0 - for success, 1 - for abort
  \item Git finish transaction if the hook exited with 0 status, or cancel it on abort; after that it
    calls hook again to notify completion
  \item Hook notify RM process about transaction finish
  \item RM notifies TM coordinator about finish transaction
\end{itemize}

To evaluate the 3PC protocol in this workflow, we've created a $TLA_{+}$ model and checker
(todo: create the model, add to appendix).

It shows that 3PC in git repository transaction could be broken by these paths: (todo: describe paths).

\section{Conclusion}

(todo: analyze the result of paths and show the risks how it affects git repository consistency.)

\end{document}
