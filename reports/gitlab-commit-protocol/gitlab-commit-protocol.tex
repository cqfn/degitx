\documentclass[acmlarge, screen, nonacm]{acmart}
\usepackage[utf8]{inputenc}
\usepackage{float}
\usepackage{setspace}
\usepackage{pgfplots}
\usepackage{graphicx}
\usepackage{fancyvrb}
\usepackage{listings}
\usepackage{xcolor}
\usepackage{hyperref}
  \hypersetup{colorlinks=true,allcolors=blue!40!black}
\setlength{\topskip}{6pt}
\setlength{\parindent}{0pt} % indent first line
\setlength{\parskip}{6pt} % before par
% custom commands

\newcommand{\code}[1]{\texttt{#1}}
\newcommand{\todo}[1]{\textcolor{red}{TODO: #1}}

\title{GitLab commit protocol implementation}

\author{Ivan Bukhtiyarov}
\email{van4elotti@gmail.com}

\author{Kirill Chernyavskiy}
\email{g4s8.public@gmail.com}

\acmBooktitle{none}
\acmConference{none}
\editor{none}

\begin{document}

\begin{abstract}
  Current Gitaly Cluster architecture requires preventing unexpected data loss and organizing strong consistency (
  \emph{\href{https://gitlab.com/groups/gitlab-org/-/epics/1189}{Gitaly Cluster: strong consistency}}).
  Due to experiments that are related to this problem
  \emph{\href{https://gitlab.com/gitlab-org/gitaly/-/issues/2466}{Implement 3PC for WriteRef RPC}},
  \emph{\href{https://gitlab.com/gitlab-org/gitaly/-/issues/2529}{3PC git-update-ref experiment}} and
  \emph{\href{https://gitlab.com/gitlab-org/gitaly/-/issues/2635}{2PC via pre-receive hook}}
  it was suggested that GitLab probably uses three-phase commit (3PC) protocol for managing atomic transactions. 
  But 3PC and any other commit protocols has many different implementations and
  additional extensions (such as recovery and termination protocols): it means that protocols could
  be implemented differently depends on business requirements. In this report we analyze what
  components of commit protocol is used by GitLab and how it achieves (or doesn't achieve) 
  strong consistency, by describing the whole workflow of each git push command from 
  client to replicas and investigating extension protocols for dealing with node and network failures.
\end{abstract}

\maketitle

\section{Introduction}
  Achieving strong consistency in gitaly is dedicated to an
  \emph{\href{https://gitlab.com/groups/gitlab-org/-/epics/1189}{epic}}. 
  that includes a set of issues. Some of tickets were successfully implemented and integrated,
  some of them remained on POC stage or simply closed.
  
  Curent Gitaly Cluster allows Git repositories to be replicated on multiple Gitaly nodes. 
  This improves fault tolerance by removing single points of failure. 
  Reference transactions causes changes to be broadcast to all the Gitaly nodes in the cluster, 
  but only the Gitaly nodes that vote in agreement with the primary node persist the changes to disk. 
  If all the replica nodes dissented, only one copy of the change would be persisted to disk, 
  creating a single point of failure until asynchronous replication completed.

  Quorum-based voting improves fault tolerance by requiring a majority of nodes 
  to agree before persisting changes to disk. Dissenting nodes are automatically brought in sync by
   asynchronous replication from the nodes that formed the quorum.

  As for \emph{\href{https://gitlab.com/groups/gitlab-org/-/epics/1189}{praefect}}, it can configured with one of two types of consistency:
  \begin{description}
    \item[Eventual concistency]
    Praefect guarantees eventual consistency by replicating all writes 
    to secondary nodes after the write to the primary Gitaly node has happened.
    \item[Strong concistency]
    Praefect can instead provide strong consistency by creating a transaction and writing
     changes to all Gitaly nodes at once. If enabled, transactions are only available for a subset of RPCs. 
  \end{description}

  \todo{the introduction is too overloaded with versatile information need to refactor the report with sutable sections}
\section{Workflow}

TODO: the workflow of GitLab push: from client to Gitaly nodes.

\section{3PC}

Due to current status of code research and the facts from these 
(\emph{\href{https://gitlab.com/gitlab-org/gitaly/-/issues/2466}{Implementation 3PC for WriteRef RPC}},
\emph{\href{https://gitlab.com/gitlab-org/gitaly/-/issues/2529}{3PC git-update-ref experiment}})
POCs we can assume that the three-phase commit is not implemented in gitlab at all.

3PC for WriteRef was suggested in this
\emph{\href{https://gitlab.com/gitlab-org/gitaly/-/issues/2466}{MR}},
but it was not merged to actual branches. (All the reasons are in the process of clarification)

TODO: what version of 3PC is used for GitLab. How GitLab achieves atomic commit protocol properties:
stability, consistency, non-triviality, non-blocking described in
``Consensus on transaction commit by J.Gray and L.Lamport''.

\section{Recovery protocol}


\emph{\href{https://gitlab.com/gitlab-org/gitaly/-/blob/master/internal/praefect/nodes/sql_elector.go}{sqlElector }}manages the primary election for one virtual storage (aka
shard). It enables multiple, redundant Praefect processes to run,
which is needed to eliminate a single point of failure in Gitaly High
Avaiability.

The sqlElector is responsible for:
\begin{itemize}
\item Monitoring and updating the status of all nodes within the shard.
\item Electing a new primary of the shard based on the health.
\end{itemize}
Also very Praefect node periodically performs a health check RPC with a Gitaly node.  Once the health checks are complete, Praefect node does a \textit{SELECT} from \textit{node\_status} table to determine healthy nodes.

The primary of each shard is listed in the
\textit{shard\_primaries}. If the current primary is in the healthy
node list, then \textit{sqlElector} updates its internal state to match.

Otherwise, if there is no primary or it is unhealthy, any Praefect node
can elect a new primary by choosing candidate from the healthy node
list. If there are no candidate nodes, the primary is demoted by setting the \textit{demoted} flag in \textit{shard\_primaries}.

In case of a failover, the virtual storage is marked as read-only until writes are manually enabled
again. This status is stored in the \textit{shard\_primaries} table's \textit{read\_only }column. If it is set, mutator RPCs against the storage shard should be blocked in order to prevent new primary from
diverging from the previous primary before data recovery attempts have been made.

The different scenarios of node failures described \emph{\href{https://gitlab.com/gitlab-org/gitaly/-/blob/master/internal/praefect/nodes/sql_elector.go\#L401}{in the code comments}}.

\todo{figure out the process of prefect (TM) backup}

\section{Termination protocol}

TODO: what is the termination strategy to finish the transaction.

\section{Misc}
\todo{to avoid this section (move facts to another section if it is possible)}
  Was noticed
  \emph{\href{https://gitlab.com/gitlab-org/gitaly/-/issues/2466}{Schedule replication jobs if no transactional vote happened}},
  for clarification if it will be usefull.

TODO: add references (see bibtex).
\end{document}
