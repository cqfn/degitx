\documentclass[12pt,oneside]{article}
\usepackage[utf8]{inputenc}
\usepackage{float}
\usepackage[bottom]{footmisc}
\usepackage{bookmark}
\usepackage{microtype}
\usepackage{amsmath}
\usepackage{multicol}
\usepackage{mdframed}
\usepackage{setspace}
\usepackage{pgfplots}
\usepackage{graphicx}
\usepackage{fancyvrb}
\usepackage[absolute]{textpos}\TPGrid{16}{16}
\usepackage{tikz}
  \usetikzlibrary{shapes}
  \usetikzlibrary{arrows.meta}
  \usetikzlibrary{arrows}
  \usetikzlibrary{shadows}
  \usetikzlibrary{trees}
  \usetikzlibrary{fit}
  \usetikzlibrary{calc}
  \usetikzlibrary{positioning}
  \usetikzlibrary{decorations.pathmorphing}
\usepackage{./tikz-uml}
\usepackage{everypage}
  \AddEverypageHook{
    \begin{textblock}{0.5}[0,0](0,0)
      \tikz \node[fill=myred,minimum width=0.5\TPHorizModule,minimum height=16\TPVertModule] {};
    \end{textblock}
    \begin{textblock}{0.125}[0,0](0.5,0)
      \tikz \node[fill=myblack,inner sep=0, minimum width=0.125\TPHorizModule,minimum height=16\TPVertModule] {};
    \end{textblock}
  }
\usepackage{xcolor}
  \definecolor{firebrick}{HTML}{B22222}
  \definecolor{myred}{HTML}{CF0A2C}
  \definecolor{myblack}{HTML}{232527}
\newcommand\dd[1]{\colorbox{gray!30}{\texttt{#1}}}
\usepackage{hyperref}
  \hypersetup{colorlinks=true,allcolors=blue!40!black}
\setlength{\topskip}{6pt}
\setlength{\parindent}{0pt} % indent first line
\setlength{\parskip}{6pt} % before par
% \let\oldsection\section\renewcommand\section{\newpage\oldsection}
\date{\small\today}
\title{%
  Distributed git repository manager\\
  \colorbox{firebrick}{\small\sffamily\color{white}{White Paper}}}
\usepackage[style=authoryear,sorting=nyt,backend=biber,
  hyperref=true,abbreviate=true,
  maxcitenames=1,maxbibnames=1]{biblatex}
  \renewbibmacro{in:}{}
  \addbibresource{books.bib}
\tikzset{node distance=1.6cm, auto, every text node part/.style={align=center, font={\sffamily\small}}}
\tikzstyle{block} = [draw=myblack, fill=white, inner sep=0.3cm, outer sep=0.1cm, thick]
\tikzstyle{ln} = [draw, ->, very thick, arrows={-triangle 90}, every text node part/.append style={font={\sffamily\scriptsize}}]
\begin{document}
\raggedbottom

\maketitle
\begin{abstract}
Big software compines may have millions of code repositories,
and use them extensively by programmers and CI pipelines.
One git server is not able to satisfy performance expectations,
many servers with load-balancing can't solve this issue too because
of inability storage IO scaling.
Also, a big company may have distributed teams around the world,
where each team collaborates with others in one git repo.
Cross-region repo access could be slow in such cases.
The solution for this problem is distributed git repository manager
which replicates active repositories across region nodes on updates.
\end{abstract}

% \onehalfspace

\section{The problem}

Repository user's activity is not distributed uniformly over the time.
Usually, each team uses pull or merge requests to update `master` branch,
master branch can be updated only after review, reviewers may approve multiple
pull requests in short amoun of time, and merge it into `master` branch.
After merge CI starts a new build for each new merge commit, it include repository cloning.
It produces peaks of load for git repository with frequent writes and reads IO operations.
Because of frequent writes, distributed teams can't use region replicas of repository
and going to master repo to perform `fetch`. For big repossitories and teams it may
reaches master repository IO thresholds and lead to master repo shutdown.

Merge is not the only way to "update" the repository. Lots of activities can do that,
here is the list of some:
\begin{description}
  \item[Branch create/delete]
    Create or delete branches by git push or by using web page.
  \item[Tag create/delete]
    Create or delete tags by git push or by releasing new version on web page.
  \item[Branch update]
    Developer can update feature branches byself, it can also triggers CI builds.
  \item[Tag update]
    Tag could be updated using git push.
  \item[Special REF create/update]
    On GitLab, when new merge request is created, the new `refs/merge-requests/IID/head` named ref
    is created. When source branch of merge request is updated, the ref is also updated.
  \item[Migration]
    Sometimes repository administrator can migrate source code to another phisical device, it also
    could be treated as an update.
\end{description}

\section{Requirements}
\label{sec:requirements}

\subsection{Features}
\label{sec:features}

The most critical
\href{https://en.wikipedia.org/wiki/Non-functional_requirement}{Non-functional requirements}
are:

\begin{description}
  \item[Read scalability]
    The solution should scale out the read capacity of a system, each region should be able
    to access repository without connecting to master repo frequently.
  \item[Strong consistency]
    All active replica repositories should be synchronized on updates in master
    with immediate consistency.
  \item[Self cloning]
    A node shold be able to find and synchronize new repository on read,
    after that it should be up to date on new updates.
  \item[Reliability]
    The system must have enough replicas to recover itself in case of curruption.
    Corrupted repository is reposnsible to recover itself using replica nodes.
  \item[Self management]
    Each node performs cleanup when needed (using `git gc`) and may remove replica
    from storage on read inactivity for some time.
  \item[Migration]
    Node administrator can change the storage and perform data migration from one storage
    to another.
\end{description}

\subsection{Functional Requirements}
\label{sec:nfr}

The most important \href{https://en.wikipedia.org/wiki/Functional_requirement}{functional requirements} are:

\begin{description}
  \item[Front end]
    The system potentically may have different kinds of front-ends,
    but it's required to support gRPC of GitLab to integrate the system
    into GitLab service and replace \href{https://docs.gitlab.com/ee/administration/gitaly/}{Gitaly}.
  \item[Back end]
    Each node may be connected to different types of storage for git repos,
    but it's required to support file-system storage.
  \item[Access audit]
    Node doesn't perform access control operations, but logs all
    requests with identity and performed operation.
  \item[Control mechanism]
    Repository administrators are able to add or delete node for repository and
    get all nodes status for repository.
  \item[Analytics]
    Node collects statistics for each repository and usage metrics, such as
    push and pull operations, etc. The system keeps the whole statistics about
    nodes, e.g. how many nodes contains each repository, the state of nodes, etc.
\end{description}

\subsection{Expected Metrics}
\label{ref:metrics}

In a large enterprise it is expected to have the following
numbers, in terms of load, size, and speed:

\begin{tabular}{ll}
  Repositories & 2M \\
  Active users & 100K/day \\
  Merges & 100K/day \\
  Fetches & 15M/day, 15K/m - peak \\
  Push & 200K/day \\
  Traffic - download & 200Tb/day \\
  Traffic - update & 250Gb/day \\
\end{tabular}

\printbibliography%
\end{document}
